\newcommand{\blockSpaceOne}{\vspace{1.3cm}}
\newcommand{\interBlockSpaceOne}{\vspace{1.5cm}}
\newcommand{\blockSpaceTwo}{\vspace{1.3cm}}
\newcommand{\interBlockSpaceTwo}{\vspace{.95cm}}
\newcommand{\secondBlockImSpace}{\vspace{.25cm}}
\newcommand{\thirdBlockImSpace}{\vspace{1.125cm}}

\begin{frame} % The whole poster is enclosed in one beamer frame
	\vspace{-.5cm}
	\begin{columns}
		\begin{column}{.005\textwidth}\end{column}

		
		\begin{column}{0.99\textwidth}
			\begin{columns}[t]

				\begin{column}{.01\textwidth}\end{column}

				\begin{column}{0.32\textwidth}


					\begin{block}{\large 1 Probabilistic models}
					\blockSpaceOne


					\justifying
					\begin{adjustwidth}{1.cm}{1.cm}
					 The noisy behavior of neurons is very likely theory hallmark of a stochastic computation scheme.
					 Such a scheme explains how the brain deals with ambiguous input, and how visual illusions like bi-stable images (duck/rabbit) might form, i.e., through sampling of modes.
					 According to the \textbf{neural sampling hypothesis} \cite{fiser2010statistically} certain cortical areas implement \textbf{sampling based probabilistic inference}.

					\end{adjustwidth}

					\vspace{1.cm}
					\begin{center}
						\includegraphics[width=0.7\textwidth]{example-image-a}
					\end{center}

					\begin{adjustwidth}{1cm}{1cm}
					
					These models are of particular interest for physical model systems as the brain faces similar challenges.
					\textbf{We review two works \cite{dold2019stochasticity,kungl2019accelerated}, in which we deployed stochastic spiking networks as robust models on neuromorphic hardware.}

					\end{adjustwidth}

					\blockSpaceOne
					\end{block}


					\interBlockSpaceOne



					\begin{block}{\large 2 Sampling with spikes}
					\blockSpaceOne

					\begin{adjustwidth}{1.cm}{1.cm}
					\justifying
					For applications, we require models that explicitly treat and use the sampling nature of neural networks.

					\vspace{.5cm}
					\begin{center}
						\includegraphics[width=.9\textwidth]{figTHfull}
					\end{center}
					\vspace{.5cm}

					In the \textbf{LIF sampling} framework \cite{petrovici2016stochastic} a single neuron describes a binary random variable based on its spiking behavior (Fig-A-B).
					The network approximately samples from a Boltzmann distribution over binary random variables:
					\begin{equation}
					p(z) \sim \frac{1}{Z} \exp \left ( \frac{1}{2} \sum_{ij} W_{ij} z_i z_j + \sum_{i} b_i z_i  \right)
					\end{equation}
					\end{adjustwidth}
					For practical applications we use a \textbf{hierarchical sampling network} (Fig-C) inspired by restricted Boltzmann machines \cite{hinton1984boltzmann}.



					\blockSpaceOne
					\end{block}


				\end{column}

				\begin{column}{.01\textwidth}\end{column}

				\begin{column}{0.32\textwidth}


					\begin{block}{\large 3 We may need no noise}
					\blockSpaceOne

					\begin{adjustwidth}{1.cm}{1.cm}
					\justifying
					In most models, the temporal variability experiences by the neurons is replaced by white Gaussian noise.
					This is however problematic because 1) a the \textbf{background activity of other brain areas is not necessarily a white noise} and 2) a neuromorphic implementation would require sufficient noise sources. 
					\end{adjustwidth}

					\secondBlockImSpace
					\begin{center}
						\includegraphics[width=.7\textwidth]{example-image-a}
					\end{center}
					\secondBlockImSpace

					\begin{adjustwidth}{1.cm}{1.cm}
					\justifying

					We  found \cite{dold2019stochasticity} that  \textbf{an  ensemble  of  dynamically  fully  deterministic,  but functionally probabilistic networks}, can learn a connectivity pattern that enables probabilistic computation with a degree of precision that matches the one attainable with idealized,  perfectly  stochastic  components.
					The  key  element  of  this  construction  is  self-consistency,  in  that  all input activity seen by a neuron is the result of output activity of other neurons that fulfill a functional role in their respective subnetworks
					
					\end{adjustwidth}

					\blockSpaceOne
					\end{block}

					\interBlockSpaceOne

					
				\begin{block}{\large 4 The BrainScaleS system}
					\blockSpaceOne


					\justifying
					\begin{adjustwidth}{1.cm}{1.cm}
					 On a single module of the \textbf{BrainScaleS \cite{schemmel2010wafer} analog neuromorphic hardware} (Fig-A) the physical model of \textbf{200k neurons and 40 million synapses} is implemented using CMOS technology.
					 The system follows the principle of \textbf{physical modeling}: it uses the dynamics of the underlying substrate to implement computation.
					 
					\end{adjustwidth}

					\vspace{1.cm}
					\begin{center}
						\includegraphics[width=\textwidth]{figHWfull}
					\end{center}

					\begin{adjustwidth}{1cm}{1cm}
					As such it can emulate networks of spiking neurons with \boldmath{$10^4$}\textbf{-fold speed-up} compared to biological real-time, but suffers from the variability of the parameters (Fig-B-C).
					Hence, we require robust network dynamics and learning rules.

					\end{adjustwidth}

					\blockSpaceOne
					\end{block}




				\end{column}

				\begin{column}{.01\textwidth}\end{column}

				\begin{column}{0.32\textwidth}

					\begin{block}{\large 5 Use-case on neuromorphic hardware}
					\blockSpaceTwo

					\begin{center}
						\includegraphics[width=1.\textwidth]{figInference}
					\end{center}
					\thirdBlockImSpace

					\begin{adjustwidth}{1.cm}{1.cm}
					\justify
					 Using the LIF sampling framework we implemented a \textbf{restricted Boltzmann machine} (RBM) \cite{hinton1984boltzmann} on the BrainScaleS System \cite{kungl2019accelerated}.
					 We evaluate the model on a reduced version of the MNIST dataset.
					 The original pictures were binarized, reduced to $12\times12$ pixels and the digits 0,1,4 and 7 were selected (Fig-A).
					\end{adjustwidth}

					\thirdBlockImSpace
					\begin{center}
						\includegraphics[width=1.\textwidth]{figPatternComp}
					\end{center}
					\thirdBlockImSpace


					\begin{adjustwidth}{1.cm}{1.cm}
					\justify
					We use an on the host computer pretrained RBM and perform \textit{in-the-loop training } to compensate for the model and substrate imperfections.
					The \textbf{classification} rate recovers software level performance after $O(10)$ training steps (Fig-B).
					The implemented model is able to \textbf{complete partially occluded images} while predicting the label correctly (Fig-C-F).
					Finally it is able to \textbf{generate recognizable images} if the respective label is clamped (Fig-G).

					\end{adjustwidth}
					\thirdBlockImSpace
					\begin{center}
						\includegraphics[width=1.\textwidth]{figTsne}
					\end{center}

					
					\blockSpaceTwo
					\end{block}


					\interBlockSpaceTwo




				\end{column}

				\begin{column}{.005\textwidth}\end{column}


			\end{columns}



		\end{column}
		\begin{column}{.005\textwidth}\end{column}
	\end{columns}

	\vspace{0cm}

	\begin{columns}[t]
		\begin{column}{.01\textwidth}\end{column}

		\begin{column}{.98\textwidth}
			\begin{block}{References}
			 \begin{minipage}{0.79\linewidth}
								\tiny
								\bibliographystyle{ieeetr}
								\bibliography{bib}
			 \end{minipage}

			\end{block}
		\end{column}

		\begin{column}{.01\textwidth}\end{column}
	\end{columns}

\end{frame} % End of the enclosing frame
