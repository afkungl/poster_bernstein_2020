\documentclass{standalone}
\usepackage{tikz}
\usetikzlibrary{arrows,arrows.meta,positioning}
\usetikzlibrary{calc}
\usetikzlibrary{shapes.geometric}

\begin{document}
\begin{tikzpicture}[
    font=\sffamily,
    line width=1pt,
    scale=1.0,
    shorten >=1pt,
    shorten <=1pt,
    transform shape,
    vertex/.style={circle, fill=black!30, draw, minimum size=.4cm, inner sep=0pt},
    vertexLarge/.style={circle, draw, minimum size=1.5cm, inner sep=0pt},
    syn/.style={->,> = latex',draw=black,fill=black,line width=1.5pt,bend left=10}, % <<<<<<<<<<<
    exc/.style={->,> = latex',draw=red,fill =red,line width=.5pt,bend left=10}, % <<<<<<<<<<<
    excFadeOut/.style={> = latex',draw=red,fill =red,line width=.5pt},
    inhFadeOut/.style={> = latex',draw=blue,fill =blue,line width=.5pt},
    inh/.style={->,> = latex',draw=blue,fill=blue,line width=.5pt,bend left=25},
    excB/.style={->,> = latex',draw=red,fill =red,line width=.5pt,bend right=25}, % <<<<<<<<<<<
    inhB/.style={->,> = latex',draw=blue,fill=blue,line width=.5pt,bend left=25},
    squareI/.style={regular polygon,regular polygon sides=4, draw=blue, fill=blue, minimum size=.1cm, inner sep=0pt},
    squareE/.style={regular polygon,regular polygon sides=4, draw=red, fill=red, minimum size=.1cm, inner sep=0pt},
    excE/.style={-,draw=red,fill =red,line width=.3pt},
    excI/.style={-,draw=blue,fill =blue,line width=.3pt},
    inhNoise/.style={->,> = latex',draw=blue,fill=blue,line width=1.pt,bend right=20},
    excNoise/.style={->,> = latex',draw=red,fill =red,line width=1.pt,bend left=20},
    inhLoop/.style={->,> = latex',draw=blue, fill=blue ,line width=3.5pt,out = 45, in = 135, looseness= 5},
    visible/.style={circle, fill=green, draw, minimum size=.4cm, inner sep=0pt},
    hidden/.style={circle, fill=blue, draw, minimum size=.4cm, inner sep=0pt},
    label/.style={circle, fill=red, draw, minimum size=.4cm, inner sep=0pt},
    weight/.style={draw=black,fill = black,line width=1pt},
    assoc/.style={draw=gray,fill = black,line width=1pt, dotted},       
    zoom/.style={line width=.5pt,shorten >=0pt,shorten <=0pt,densely dashed},
    sonSynapse/.style={->,> = latex',draw=blue,fill=blue,line width=.5pt}   
    ]
    
    % Panel A
    
    % network of five neurons (placed on a circle)
    \begin{scope}[shift={(0.,.1)}, scale=.7]
    \begin{scope}[shift={(-3.0, -1.9)}, scale=.6, local bounding box=networkPoisson]
    \foreach \i in {1,...,5}{
        \node[vertex] (n\i) at ({(\i - 2)*360./5.}:1.) {};
    }
    
    % synapses
    \draw[weight] (n1) edge (n2);
    \draw[weight] (n2) edge (n3);
    \draw[weight] (n4) edge (n5);
    \draw[weight] (n5) edge (n1);
    \draw[weight] (n3) edge (n5);
    \draw[weight] (n1) edge (n3);
    
    
    \end{scope}
    
    % close up of one neuron with Poisson noise and functional synapses and bias neuron
    \begin{scope}[shift={($(networkPoisson.west)+(-3.075,-0.3)$)}, scale=1.]
        
        % coordinates of rectangle around close up
        \coordinate (cu1) at (-1.0,-1.1);
        \coordinate (cu2) at (2.2,1.8);
        
        % rectangle around neuron in network
        \coordinate (nw1) at (2.97,0.43);
        \coordinate (nw2) at (2.97+0.45,0.43+0.45);
        \draw[line width=.5pt] (nw1) rectangle (nw2);
        
    
        % lines relating close up to neuron in network
        \draw[zoom] (cu1) -- (nw1);
        \draw[zoom] (cu2) -- (nw2);
        \draw[zoom] (cu1-|cu2) -- (nw1-|nw2);
        \draw[zoom] (cu1|-cu2) -- (nw1|-nw2);
        
        % rectangle around close up
        \draw[line width=.5pt,fill=white] (cu1) rectangle (cu2);
        
        % sampling neuron
        \node[vertex] (s1) at(0.,0.) {\textbf{s}};
        
        % noise sources
        \node[squareE,right] (s1e) at (0.7, 0.2) {};
        \node[right=0.0 of s1e] {$\nu_\mathrm{Poisson}^e$};
        \draw[excE] (s1e) edge (s1);
        \node[squareI,right] (s1i) at (0.7, -0.2) {};
        \node[right=0.0 of s1i] {$\nu_\mathrm{Poisson}^i$};
        \draw[excI] (s1i) edge (s1);
        
        % where the arrows representing synapses point to   
        \coordinate (d4) at (-60:1.0);
        \coordinate (d3) at (-79:1.0);
        \coordinate (d2) at (-100:1.0);
        \coordinate (d1) at (-121:1.0);
        
        
        % arrows representing synapses, fade out by dotted continuation
        \draw[inhFadeOut,dotted,shorten <=2pt] (d1) edge (s1);
        \draw[inhFadeOut,shorten <=8pt] (d1) edge[->] (s1); 
        
        \draw[excFadeOut,dotted,shorten <=2pt] (d2) edge (s1);
        \draw[excFadeOut,shorten <=8pt] (d2) edge[->] (s1); 
        
        \draw[excFadeOut,dotted,shorten <=2pt] (s1) edge (d3);
        \draw[excFadeOut,shorten >=8pt] (s1) edge (d3);
        
        \draw[inhFadeOut,dotted,shorten <=2pt] (d4) edge (s1);
        \draw[inhFadeOut,shorten >=8pt] (s1) edge (d4);
        
        \node (w network) at (1.2, -0.8) {$w_\mathrm{network}$};
        
        % bias neuron
        \node[vertex,] (b1) at (0,1.4) {\textbf{b}};
        \node[right=0.0 of b1] {$\nu_b$};
        
        \draw[excB] (b1) edge (s1);
        \draw[inhB] (b1) edge node[midway,xshift = 15.0] {$w_\mathrm{bias}$} (s1) ;
    
    \end{scope}
    \end{scope}
    
    % Panel C
    
    % network of five neurons (placed on a circle)
    \begin{scope}[shift={(0.,.1)}, scale=.7]
    \begin{scope}[shift={(-3.0, 1.6)}, scale=.6, local bounding box=networkDN]
        \foreach \i in {1,...,5}{
            \node[vertex] (n\i) at ({(\i - 2)*360./5.}:1.) {};
        }
        
        % synapses
        \draw[weight] (n1) edge (n2);
        \draw[weight] (n2) edge (n3);
        \draw[weight] (n4) edge (n5);
        \draw[weight] (n5) edge (n1);
        \draw[weight] (n3) edge (n5);
        \draw[weight] (n1) edge (n3);
    
    
    \end{scope}
    
    % close up of one neuron with sea of noise noise and functional synapses and bias neuron
    \begin{scope}[shift={($(networkDN.west)+(-3.075,-0.3)$)}, scale=1.]
    
        % coordinates of rectangle around close up
        \coordinate (cu1) at (-1.0,-1.1);
        \coordinate (cu2) at (2.2,1.8);
        
        % rectangle around neuron in network
        \coordinate (nw1) at (2.97,0.43);
        \coordinate (nw2) at (2.97+0.45,0.43+0.45);
        \draw[line width=.5pt] (nw1) rectangle (nw2);
        
        
        % lines relating close up to neuron in network
        \draw[zoom] (cu1) -- (nw1);
        \draw[zoom] (cu2) -- (nw2);
        \draw[zoom] (cu1-|cu2) -- (nw1-|nw2);
        \draw[zoom] (cu1|-cu2) -- (nw1|-nw2);
        
        % rectangle around close up
        \draw[line width=.5pt,fill=white] (cu1) rectangle (cu2);
        
        % sampling neuron
        \node[vertex] (s1) at(0.,0.) {\textbf{s}};
        
        
        % where the arrows representing sea of noise synapses point to  
        \coordinate (sond1) at (9:1.0);
        \coordinate (sond2) at (-9:1.0);
        \coordinate (sond3) at (27:1.0);
        \coordinate (sond4) at (-27:1.0);
        
        % arrows representing sea of noise synapses, fade out by dotted continuation
        \draw[inhFadeOut,dotted,shorten <=2pt] (sond1) edge (s1);
        \draw[inhFadeOut,shorten <=8pt] (sond1) edge[->] (s1);  
        \draw[inhFadeOut,dotted,shorten <=2pt] (sond4) edge (s1);
        \draw[inhFadeOut,shorten <=8pt] (sond4) edge[->] (s1);  
        
        \draw[excFadeOut,dotted,shorten <=2pt] (sond2) edge (s1);
        \draw[excFadeOut,shorten <=8pt] (sond2) edge[->] (s1);  
        \draw[excFadeOut,dotted,shorten <=2pt] (sond3) edge (s1);
        \draw[excFadeOut,shorten <=8pt] (sond3) edge[->] (s1);  
        
        \node[right=0.175 of (0:0.85)] {DN};
        
        
        % where the arrows representing functional synapses point to    
        \coordinate (d4) at (-60:1.0);
        \coordinate (d3) at (-79:1.0);
        \coordinate (d2) at (-100:1.0);
        \coordinate (d1) at (-121:1.0);
        
        
        % arrows representing functional synapses, fade out by dotted continuation
        \draw[inhFadeOut,dotted,shorten <=2pt] (d1) edge (s1);
        \draw[inhFadeOut,shorten <=8pt] (d1) edge[->] (s1); 
        
        \draw[excFadeOut,dotted,shorten <=2pt] (d2) edge (s1);
        \draw[excFadeOut,shorten <=8pt] (d2) edge[->] (s1); 
        
        \draw[excFadeOut,dotted,shorten <=2pt] (s1) edge (d3);
        \draw[excFadeOut,shorten >=8pt] (s1) edge (d3);
        
        \draw[inhFadeOut,dotted,shorten <=2pt] (d4) edge (s1);
        \draw[inhFadeOut,shorten >=8pt] (s1) edge (d4);
        
        \node (w network) at (1.2, -0.8) {$w_\mathrm{network}$};
        
    
        
        % bias neuron
        \node[vertex,,] (b1) at (0,1.4) {\textbf{b}};
        \node[right=0.0 of b1] {$\nu_b$};
        
        \draw[excB] (b1) edge (s1);
        \draw[inhB] (b1) edge node[midway,xshift = 15.0] {$w_\mathrm{bias}$} (s1) ;
        
    \end{scope}
    
    % sea of noise network (more precisely: decorrelation network) (network of five neurons (placed on a circle))
    \begin{scope}[shift={($(networkDN.west)+(2.7,0.)$)}, scale=.6, local bounding box=DNnetwork]
        \foreach \i in {1,...,5}{
            \node[vertex] (son\i) at ({(\i - 2)*360./5.}:1.) {};
        }
        
        % synapses
        \draw[sonSynapse] (son1) edge (son2);
        \draw[sonSynapse] (son3) edge (son2);
        \draw[sonSynapse] (son5) edge (son1);
        \draw[sonSynapse] (son2) edge (son4);
        \draw[sonSynapse] (son3) edge (son4);
        \draw[sonSynapse] (son2) edge (son5);
%       \foreach \i in {1,...,5}
%           \foreach \j in {1,...,5}{
%               \ifthenelse{\NOT \i = \j }{
%                   \draw[draw=blue,line width=1pt] (son\i) edge (son\j);
%               }
%               {}
%           
%       }

        
        
        % circle around sea of noise
        \node[vertexLarge,minimum size=80.] (Son) at (0.0,0.0) {};
        %\draw[inhLoop] (Son) edge node[above=.2em] {inh.\ recurrent} (Son);
        \draw[excNoise,shorten >=0pt] (Son) edge node[below] {} (-2.6,-.5);
        \draw[inhNoise,shorten >=0pt] (Son) edge node[above] {} (-2.6,.5);
    
    
    \end{scope}
    \end{scope}
    
    % DN network in normal font size
    \node[above = 0. of DNnetwork]{\hspace{0.7cm}DN};
         

    \pgfresetboundingbox
    \draw[use as bounding box,inner sep=0pt] node {\includegraphics[width=\textwidth]{figSetup}};
    \end{tikzpicture}
\end{document}